\section{Introduction}

Recent experiments reveal the existence of a fine connectivity structure in cortical networks. The local network architecture of pyramidal cells in the cortex are significantly different from an Erdös-Rényi network. The pyramidal neurons in the cortex have been found to be connected to each other in a non-random fashion forming structured local cortical circuits. The experimental data show that there are significant number of reciprocal connections between the pyramidal neurons and that there is a prevalence of other motifs involving groups of three or more highly interconnected neurons \cite{markram1997,thomson2002, Song2005, Perin2011}. Fast spiking interneurons and pyramidal cells are reciprocally connected with probability close to one \cite{Yoshimura2005}. Together, these results provide a picture of local cortical microcircuit where (i) connectivity motifs occur greater than expected in a random network, and (ii) a very few strong connections dominate in a sea of weak synapses.\\

What effects does such a fine connectivity structure have on cortical dynamics? Here we studied the simplest type of connectivity motif: bidirectionality. We address the question by studying the effects of excess bidirectional connections in the framework of Balanced networks \cite{carl1996, carl1998, carl2004}, since this framework has been successful in explaining several ubiquitous features of cortical dynamics without resorting to fine tuning of network parameters \cite{softky1993, Holt1996, roxin2011}. These networks automatically find an operating point where total excitatory and inhibitory inputs to the neurons in the network approximately cancel or balance each other.  Hence, the effective net inputs that the neurons receive are of $O(Threshold)$. Modulations or changes in the $O(Threshold)$ inputs lead to emergence of interesting network dynamics. In this study, we are interested in investigating if bidirectionality can lead to significant $O(Threshold)$ changes in the total input to neurons, and if so, to characterize the resulting changes on the network dynamics. In the standard balanced network the connectivity is assumed to be random. Neurons in a random balanced network receive weakly time-correlated $O(Threshold)$ inputs or fast fluctuating inputs. As a result, input currents to the neurons in the network and consequently the spike time auto-correlations decorrelate rapidly\cite{carl1996, carl1998}. However, this is not consistent with what has been observed in cortical recordings, the power spectrum of membrane voltages recorded in the cortex show significant low frequency components. \cite{Tan2014}.\\	
	
We studied the effect of bidirectionality of in a two population network consisting of Excitatory(E) and Inhibitory(I) neurons. We considered bidirectionality within each population(E-to-E and, I-to-I) and between the two populations(E-to-I) to address the following questions. (i) What are the effects of excess bidirectionality on the dynamics in a balanced network? (ii) What are the effects on functional properties? (iii) What are the effects on fluctuations and spiking statistics?
We investigated these questions using numerical simulations of large scale spiking network  model of rodent V1 (see Methods). \\
 
E-to-E bidirectionality had neligible effects. I-to-I bidirectionality was efficacious in slowing down fluctuations and thereby increasing trial to trial response variability. The time scale of slow fluctuations could be well approximated by a power law for low to moderate range of excess bidrectionality. Slow synapses in addition to I-to-I bidirectionality promote multistability. And, E-to-I bidirectionality introduces negative auto-correlations in the spiking responses. 
