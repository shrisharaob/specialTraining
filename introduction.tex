\section{Introduction}
With the advent of new techniques of labeling and identifying cells, cortical wiring is no longer thought to be unorganized. Various rules governing the organization of synaptic connectivity have been proposed, such as dependency of connection probability and strength based on functional properties\cite{Ko2011, Cossell2015, Lee2016}.  Specific connections between different cell types have been identified. Several types of ordered connectivity schemes have been discovered at the level of microcircuits, such as motifs formed due to strongly connected groups of neurons, groups of neurons specifically targeted by feed forward input. The local network architecture of pyramidal cells in the cortex are significantly different from an Erdös-Rényi network. For instance, the pyramidal neurons in the cortex have been found to be connected to each other in a non-random fashion forming structured local cortical circuits. The experimental data show that there are significant number of reciprocal connections between the pyramidal neurons and that there is a prevalence of other motifs involving groups of three or more highly interconnected neurons \cite{markram1997,thomson2002, Song2005, Perin2011}. Fast spiking interneurons and pyramidal cells are reciprocally connected with probability close to one \cite{Yoshimura2005}.\\
Together, these results provide a picture of local cortical microcircuit where specific connections occur greater than expected in a random network and a very few strong connections dominate in a sea of weak synapses. What then is the role of fine structure? How does it shape the dynamics and functional properties of cortical networks? One would expect that all the effects of adding structure to a random network at the microcircuit level would average out as the number of neurons are increased.  \\

Here we studied the simplest type of fine structure i.e. bidirectionality in the framework of balanced networks \cite{carl1996, carl1998, carl2004}. This framework has been successful in explaining several ubiquitous features of cortical dynamics without resorting to fine tuning of network parameters \cite{softky1993, Holt1996, roxin2011}. These networks automatically find an operating point where total excitatory and inhibitory inputs to the neurons approximately cancel or balance each other. The activity and consequently the functional properties of the network is then dictated by the modulations in the net input that are of the same order of magnitude as the threshold. In this study, we are interested in investigating if bidirectionality can lead to significant changes in this net input . \\
	
We studied the effect of bidirectionality in a two population network consisting of Excitatory(E) and Inhibitory(I) neurons. We considered bidirectionality within each population(E-to-E and, I-to-I) and between the two populations(E-to-I) to address the following questions: What are the effects of excess bidirectionality on the on functional properties? What are its effects on the dynamics in a balanced network and on fluctuations and spiking statistics? We investigated these questions using numerical simulations of large scale spiking network  model of rodent V1 (see Methods). \\
