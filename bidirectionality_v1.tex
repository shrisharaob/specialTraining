\documentclass[10pt, a4paper, twocolumn]{article}
\usepackage[utf8]{inputenc}
\usepackage{amsmath}
\usepackage{amsfonts}
\usepackage{amssymb}
\usepackage{graphicx}
\usepackage[a4paper, total={6.5in, 8in}]{geometry}
%\usepackage{caption}
\usepackage{subcaption}
\captionsetup[subfigure]{position=top, labelfont=bf,textfont=normalfont,singlelinecheck=off,justification=raggedright}
\title{Effects of excess bidirectionality in cortical networks}
\date{}

\begin{document}
%\renewcommand{\thefigure}{S\arabic{figure}}
%\tableofcontents
\maketitle
\begin{@twocolumnfalse}
    \maketitle
    \begin{abstract}
     % E-to-E bidirectionality had neligible effects. I-to-I bidirectionality was efficacious in slowing down fluctuations and thereby increasing trial to trial response variability. The time scale of slow fluctuations could be well approximated by a power law for low to moderate range of excess bidrectionality. Slow synapses in addition to I-to-I bidirectionality promote multistability. And, E-to-I bidirectionality introduces negative auto-correlations in the spiking responses.
    \end{abstract}
  \end{@twocolumnfalse}

\section{Introduction}

Recent experiments reveal the existence of a fine connectivity structure in cortical networks. The local network architecture of pyramidal cells in the cortex are significantly different from an Erdös-Rényi network. The pyramidal neurons in the cortex have been found to be connected to each other in a non-random fashion forming structured local cortical circuits. The experimental data show that there are significant number of reciprocal connections between the pyramidal neurons and that there is a prevalence of other motifs involving groups of three or more highly interconnected neurons \cite{markram1997,thomson2002, Song2005, Perin2011}. Fast spiking interneurons and pyramidal cells are reciprocally connected with probability close to one \cite{Yoshimura2005}. Together, these results provide a picture of local cortical microcircuit where (i) connectivity motifs occur greater than expected in a random network, and (ii) a very few strong connections dominate in a sea of weak synapses.\\

What effects does such a fine connectivity structure have on cortical dynamics? Here we studied the simplest type of connectivity motif: bidirectionality. We address the question by studying the effects of excess bidirectional connections in the framework of Balanced networks \cite{carl1996, carl1998, carl2004}, since this framework has been successful in explaining several ubiquitous features of cortical dynamics without resorting to fine tuning of network parameters \cite{softky1993, Holt1996, roxin2011}. These networks automatically find an operating point where total excitatory and inhibitory inputs to the neurons in the network approximately cancel or balance each other.  Hence, the effective net inputs that the neurons receive are of $O(Threshold)$. Modulations or changes in the $O(Threshold)$ inputs lead to emergence of interesting network dynamics. In this study, we are interested in investigating if bidirectionality can lead to significant $O(Threshold)$ changes in the total input to neurons, and if so, to characterize the resulting changes on the network dynamics. In the standard balanced network the connectivity is assumed to be random. Neurons in a random balanced network receive weakly time-correlated $O(Threshold)$ inputs or fast fluctuating inputs. As a result, input currents to the neurons in the network and consequently the spike time auto-correlations decorrelate rapidly\cite{carl1996, carl1998}. However, this is not consistent with what has been observed in cortical recordings, the power spectrum of membrane voltages recorded in the cortex show significant low frequency components. \cite{Tan2014}.\\	
	
We studied the effect of bidirectionality of in a two population network consisting of Excitatory(E) and Inhibitory(I) neurons. We considered bidirectionality within each population(E-to-E and, I-to-I) and between the two populations(E-to-I) to address the following questions. (i) What are the effects of excess bidirectionality on the dynamics in a balanced network? (ii) What are the effects on functional properties? (iii) What are the effects on fluctuations and spiking statistics?
We investigated these questions using numerical simulations of large scale spiking network  model of rodent V1 (see Methods). \\
 
E-to-E bidirectionality had neligible effects. I-to-I bidirectionality was efficacious in slowing down fluctuations and thereby increasing trial to trial response variability. The time scale of slow fluctuations could be well approximated by a power law for low to moderate range of excess bidrectionality. Slow synapses in addition to I-to-I bidirectionality promote multistability. And, E-to-I bidirectionality introduces negative auto-correlations in the spiking responses. 

\section{Results}
Orientation selectivity(OS) has been studied extensively in cat and monkey since its discovery by Hubel and Wiesel in the primary visual cortex(V1). When presented with oriented bar stimuli, the response of the a V1 neuron increases when the bar is optimally oriented in a certain direction. In these animals, anatomically proximate neurons in the V1 have very similar preferred orientation forming orientation maps. Orientation maps  OS neurons are neatly arranged to from pinwheel-like patterns. Each pinwheel has a singularity at the center and orientation domains away from the center. Assuming connection probabilities drop with anatomical distance, neurons receive inputs mostly from neurons of the similar PO. It has been proposed that strong OS is possible only in the presence of functional connectivity. A functional connectivity where neurons with similar orientation preference have a higher probability of connection.\\
However, strong OS has also been observed in rats, mice and squirrels. Neurons in V1(L2/3) of these rodents show strong OS without orientation maps. The OS neurons in the rodent V1 are arranged in a salt and pepper fashion where the POs of neurons are interspersed forming no discernible pattern. In addition, the evidence for functional connectivity in V1 of rodents is mixed. How can strong orientation selectivity be achieved without a functional architecture? Since there is strong OS in rodent V1 in spite of an orientation map, it was it was hypothesized that rodent V1 should posses a functional architecture. Hansel and van Vreeswijik proposed an alternative solution. They showed that strong feature selectivity can be achieved in purely random networks under very general conditions. In their model, the recurrent dynamics of the network is such that the total inhibitory and excitatory currents cancel each other. Each neuron in the network receives inputs from other neurons of different POs. Hence the the total input has a large untuned component and a comparably weak tuned part. As noted earlier, the large untuned component is dynamically suppressed and the weakly tuned input is amplified by the single neuron threshold nonlinearity. Both the excitatory and inhibitory populations exhibit strong OS. In fact, this is generic result showing that a randomy connected network of neurons with strong synapses can develop feature selectivity. Which is true only when the total exitatory and inhibitory currents that each neuron receives are approximately equal. \\ 
% Recently, Hansel \& Vreeswijk \cite{Hansel2012} showed that a functional connectivity is not a necessary ingredient for the achievement of strong orientation selectivity. They showed that in a purely random network, strong orientation selectivity can arise provided that cortex in operating in the balanced State. In fact, this is generic result showing that a randomy connected network of neurons with strong synapses can develop feature selectivity. Which is true only when the total exitatory and inhibitory currents that each neuron recieves are approximately equal. \\ 
The inhibitory population in the study of Hansel \& Vreeswijk \cite{Hansel2012} had tuning properties very similar to the excitatory population. This is not compatible with recent experimental results which clearly indicate that inhibitory population(PV\textsuperscript{+}) is more broadly tuned than excitatory neurons. But, here we show that this can be explained when the inhibotory neurons(PV\textsuperscript{+}) in L2/3 recieve recieve more numerous and weaker synapses from L4 than the excitatory population. This is a prediction which has to be tested in future experiments. There is already some evidence from the barrel cortex of the mouse suggesting that such a bias in the number of feedforward inputs the excitatory and inhibitory population. Hence, in all our simulations, we assumed that the inhibory population in L2/3 revieces weaker and more numerous inputs than the excitatory population($K_{ff}^{E} = 100, \; K_{ff}^{I} = 800$). This gives us mean values of Circular variance($CircVar = 1 - OSI$) that are comparable to the exprimental numbers. 
%Although OS has been studied for over 40 years, the mechanism by which feature seletivity arises in the cortex still debated. Here is an example tuning curve from a V1 neuron of the mouse. % The neurons in the V1 of these animals are orgainzed in what is called a salt and pepper fashion, where there is no dercernable arangement of neurons according to their preferred orinetation as we had seen for the Cat. And in addtion to this fact, the evidence for fuctional connectivity in V1 of rodents is mixed. So the question that arises is 'How can strong orientation selectivity be achieved without a functional architecture?



\subsection{Numerical simulations}
We studied the effects of bidirectionality in a conductance based model of rodent V1 L2/3 developed my Hansel and van Vreeswijk \cite{Hansel2012}. 
% Neurons in L2/3 rodent V1 show strong OS which are arranged in a salt and pepper fashion. However in primates like Cat and Monkey, the neurons are arranged in a manner such that anatomically close by neurons have similar PO. Such an arrangement leads to the well documented orientaion maps. Thus, it was hypothesized that strong OS requires a functional architecture where neurons with similar OS are connected to each other. Since there is no strong OS in rodent V1 in spite of an orientation map, it was it was claimed that rodent V1 should posses functional architecture. Hansel and van Vreeswijik proposed an alternative solution. They showed that strong feature selectivity can be achieved in purely random networks under very general conditions. In their model, the recurrent dynamics of the network is such that the total inhibitory and excitatory currents cancel each other. Each neuron in the network recieves inputs from other neurons of different POs. Hence the the total input has a large untuned component and a comparably weak tuned part. As noted earlier, the large untuend component is dynamically suppressed and the weakly tuned input is amplified by the single neuron threshold nonlinearity. Both the excitatory and inhibitory populations exhibit strong OS. However, there is a growting concensus that PV$^{+}$ cells are broadly tuned in ordents. We discovered that, it is possible to explain broadly tuned inhibitory cells by incorporating more numerous and weaker feed-forward input to the inhibitory population. \\
The connectivity matrix ($N \times N$)in their study was random with each neuron in the network receiving $K$ inputs on average. In such networks, there are $\frac{K^2}{N}$ bidirectional connections just by chance. Excess bidirectionality was introduced in the network such that the number of such connections were above chance level with a probability $p$, giving on average $pK$ number of bidirectional connections. The probability $p$ can be interpreted as a symettry parameter quantifying the amount of symmetry in the connectivity matrix, the value of the symmetry parameter for symmetric random matrix would be $p = 1$ and a non-symmetric random matrix will have $p = 0$. %We study the deviation of the dynamics of the network
Excess bidirectionality was introduced in all the three possible combinations, i.e (i) within the Excitatory(E) population(E-to-E), (ii) within the Inhibitory(I) populations(I-to-I) and, (iii) betweeen Excitatory and Inhibitory populations(E-to-I). Using the modified model of Hansel \& van Vreeswijk accounting for broadly tuned inhibitory population we characterized the the spiking irregularity, fluctuations and, response properties of the L2/3 model network with excess bidirectionality. 
%The firing rate, circular variance and Fano factor distributions were computed for each 
% \textbf{•}case.\\
%\emph{Spiking irregularity and fluctuations}
Excess E-to-E bidirectionality has been observed experimentally at probabilities in the range of $p = 0.2$ to $p = 0.3$[REFerence]. When the probability is non-zero in a network, it leads to the formation of a significant number of loops in the network connectivity. Such loops can have a positive or negative feedback or self-coupling. In a balanced network, E-to-E excess bidirectionality leads to a delayed positive self-coupling. Let us consider one E neuron in a balanced network with excess bidirectional probability $p$ and each neuron receiving $K$ connections on average. Then expected number of bidirectional connections that this neuron receives is $pK$. When this neuron fires a spike, it results in an increase in the input to all the neurons it projects to by $\frac{J_{EE}}{\sqrt{K}}$, thus increasing their probability of firing. And a fraction $p$ of those neurons project back to the neuron under consideration. This leads to an increase in the probability that this neuron will fire again will increase by an amount proportional to $p J_{EE}^{2}$, which is $O(Threshold)$. 
The simulations show that E-to-E bidirectionality has negligible effects even at high levels of bidirectionality (Fig. \ref{fig:e2e0} \& Fig. \ref{fig:e2e1}). The firing rate distribution for networks with bidirectional probability as high as $p = 0.8$ show no noticeable change as it approximately overlaps with control($p = 0$, i.e random network) condition. Similarly, tuning properties(quantified using Circular variance) and Fano factor distributions of the both the excitatory and inhibitory populations remain unchanged compared to control. \\
In contrast to E-to-E reciprocity, adding I-to-I bidirectionality is much more efficacious in slowing down the fluctuations as is clearly seen in the population averaged AC function(Fig. \ref{fig:i2i0}). As a result the neurons become prone to fire in bursts.(Fig. \ref{fig:vmtrace})

At experimentally observed levels of bidirectional probabilities (which is estimated to be in the range 0.2-0.3 \cite{Song2005}), the tuning properties remain unchanged (Fig. \ref{fig:i2i1}a, Fig. \ref{fig:i2i1}b).
The population averaged Fano factor increases with increasing reciprocity(Fig. \ref{fig:i2i1}c, Fig. \ref{fig:i2i1}d). Thus, I-to-I bidirectionality slows down fluctuations and enhances response variability. However, E-to-I has the opposite effect. Fluctuations decay much faster than in a random network and Fano factor is reduced (Fig. \ref{fig:e2i0}). E-to-I bidirectionality leads to and increase in negative serial correlations in the spiking activity. As mentioned earlier, negative serial correlations reduces the Fano factor(Eq. \ref{ffcveq}). Thus,  with right probabilities of I-to-I and E-to-I reciprocity, it is possible to obtain a network where their individual effects will cancel each other. \\
We also investigated the effect of reciprocity in the presence of slow synapses, the time constants of recurrent and feed forward synapses were increased. Fig. \ref{fig:slowsyn0} shows that slow synapses and strong I-to-I bidirectionality lead to a large increase in the decorrelation times and Fano factors.\\  A fully random network is characterized by a single stable state and is independent of initial conditions. Fig. \ref{fig:slowsyn0}e shows that steady state firing rates of all the neurons converge to nearby values even though the simulations are started with different initial conditions. Slow synapses along with large I-to-I bidirectionality results in mutistability in the networks.(Fig. \ref{fig:slowsyn0}f) shows that different initial conditions leads to the network converging to different steady states. To ensure that the this was not an effect of poor firing rate estimates, these simulations were run for longer times(250s). Hence the steady state firing rates of the neurons depends on the initial conditions.
%A fully random network is characterized by a single ground state. Fig. \ref{fig:slowsyn0}e shows that steady state firing rates of all the neurons converge to nearby values even though the simulations are started with different initial conditions. On the other hand, a fully bidirectional network is very similar to a diluted spin glass. Thus a replica symmetry breaking (RSB) may be expected in this case. Therefore, as the level of bi-directionality is increased one can expect glassy behaviour to appear with a slowing down of the fluctuations. In the simulations this is seen in the presence of slow synapses at high levels of bidirectionality (Fig. \ref{fig:slowsyn0}f).
Thus, we show that excess bidirectional connections between I-to-I slow down the fluctuations in the neuronal input in balanced networks. As a result, the autocorrelation of the activity decays more slowly than in the corresponding Erdös-Rényi network. Furthermore, bidirectional connections between I cells increase the Fano factor of the spike count. These phenomena are due to the small loops that the bidirectionality induces in the network architecture. Together with the relatively strong synapses in balanced networks these lead to a non-negligible effective delayed self-coupling. On the other hand E-to-I bidirectionality reduces the decorrelation time and response variability. Slow synapses and I-to-I bidirectionality lead to a further increase in decorrelation time and promote multistability.

%Remarkably, bidirectional connections between I cells are more efficacious in slowing down the dynamics than those between E cells. 
%We show that a balanced network model of rodent V1 with conductance-based spiking neurons and excess biderctionality behaves qualitatively similarly to the binary network. 



\section{Discussion}
We show that depending on the type of bidirectional connections in balanced networks, the input decorrelation time increases compared to networks with random connectivity. The effective net inputs that the neurons receive in a balanced network are of same order of magnitude as the threshold $\left( O(Threshold) \right)$. Modulations or changes in the inputs that are $O(Threshold)$ can lead to interesting network dynamics. In this study, we were interested in investigating if bidirectionality can lead to significant $O(Threshold)$ changes in the total input to neurons, and if so, to characterize the resulting changes in functional properties and network dynamics. In the standard balanced network the connectivity is assumed to be random. So, neurons receive weakly time-correlated order of threshold inputs or fast fluctuating inputs. As a result, input currents to the neurons in the network and consequently the spike time auto-correlations decorrelate rapidly\cite{carl1996, carl1998}. \\
%However, this is not consistent with what has been observed in cortical recordings, the power spectrum of membrane voltages recorded in the cortex show significant low frequency components. \cite{Tan2014}.
The presence of excess bidirectionality leads to formation of a significant number of loops in the network connectivity. Such loops can lead to non-negligible positive or negative feedback input currents if the effect is of $O(Threshold)$. To see how such loops can contribute effects of $O(Threshold)$ let us consider one E neuron in a network with excess bidirectional probability $p$ where each neuron receives $K$ connections on average of strength $\frac{J_{EE}}{\sqrt{K}}$. Then, the expected number of bidirectional connections that this neuron receives is $pK$. When this neuron fires a spike, it results autocorrelation in an increase in the input to all the neurons it projects to by an amount proportional to $\frac{J_{EE}}{\sqrt{K}}$, thus increasing their probability of firing. And a fraction $p$ of those neurons project back to the neuron under consideration. This results in the neuron receiving an additional input proportional to $p J_{EE}^{2}$, which is $O(Threshold)$. Thus E-to-E excess bidirectionality should lead to a delayed positive self-coupling. 
%Therefore, presence of a fine structure such as bidirectionality in the network connectivity can lead to significant temporal correlations in the inputs.
%  Whereas, in a balanced network with random connectivity, the neuronal activities are weakly correlated resulting in all the neurons receiving weakly correlated inputs.\\ 
Remarkably, E-to-E bidirectionality which has been experimentally documented has negligible effects in the balanced V1 model. Why is there no effect even at excess E-to-E bidirectional probabilities close to one in our simulations?  This could be explained when we consider the balance conditions: $J_{EI} > J_{EE}$ and $\frac{J_{EI}}{J_{EE}} < \frac{J_{II}}{J_{IE}}$ \cite{carl1998}. The average E-to-E synaptic strength is smaller compared to that of average I-to-E and I-to-I, hence excess bidirectionality in E-to-E has negligible effect on the dynamics of the network. Furthermore, since I-to-I synaptic strength is strong, we hypothesized that the effects of excess bidirectionality might be more pronounced when it is introduced in the I population. So, we subsequently studied excess bidirectionality in the I population which indeed lead to longer decorrelation times.
%The slower dynamics in the network is due to temporally correlated inputs or slow fluctuations. 
The origin of these slow fluctuations is also due to delayed self-coupling, but mediated through an effective disinhibition. Self-coupling occurs due to the presence of loops in the I-to-I connectivity. Let us consider one I cell in the network, after it has fired a spike it hyperpolarizes its postsynaptic I neurons by a small amount. A fraction $pK$ of those neurons which project back to the neuron now have a slightly lower probability of firing. Hence, soon after the given neuron has spiked there is a significant reduction in the recurrent inhibition that it receives, thereby increasing the probability of its spiking soon after it has fired a spike. Therefore, the ISIs become positively correlated, such serial correlations in the spike trains leads to higher trial to trial variability. 

Bidirectionality in I-to-I affects the time scale of the dynamics only because the network is operating in the balanced regime. This could be a mechanism leading to slow fluctuations in the cortex as reported by significant power in the low frequency spectrum of membrane voltages\cite{Tan2014}. But, there is very little experimental data reporting significant I-to-I bidirectionality. Future experiments might shed more light on the connectivity scheme within the inhibitory populations.

Slow synapses and I-to-I bidrectionality leads to multistability(Fig. \ref{fig:slowsyn0}f). The emergence of multistability has been studied in Spin Glasses. A fully bidirectional network is very similar to a diluted spin glass\cite{megard1987spin}. At low temperatures, Replica Symmetry Breaking (RSB) may be expected at the transition between ferromagnetic and spin glass phases. Therefore, as the level of bi-directionality is increased one can expect glassy behaviour to appear with a slowing down of the fluctuations\cite{Crisanti1987, Crisanti1988}. The time scales of these fluctuations are typically well approximated by a power law. Although, the connectivity in SG models do not obey Dales law, one could draw an analogy between Temperature and synaptic time constant. Thus, lowering the temperature can be seen as increasing the synaptic time constant. The decorrelations times estimated from the simulations also show similar power law behaviour. (Fig. \ref{fig:slowsyn0}a \& Fig. \ref{fig:slowsyn0}b) \\

The mechanism for OS proposed by Hansel and van Vreeswijk gives rise to both excitatory and inhibitory populations with strong OS. The inhibitory population in the model had tuning properties very similar to the excitatory population. However, there is a growing consensus that PV\textsuperscript{+} cells in rodents are more broadly tuned than excitatory neurons \cite{niell2008highly, kerlin2010broadly, liu2009visual, sohya2007gabaergic, kameyama2010difference} (also see \cite{Runyan2010, Ma2010}). Here we show that this can be explained when the inhibitory neurons in L2/3 receive more numerous and weaker synapses from L4 than the excitatory population.  We discovered that, it is possible to explain broadly tuned inhibitory cells by incorporating more numerous and weaker feed-forward input to the inhibitory population. In the simulations, it was necessary to scale the synaptic strengths appropriately as the number of feedforward inputs increases to ensure that the mean population activities remain the same. Since the mean activities were fixed, broadening of the tuning of inhibitory population was not due to a change in firing rates. Recent experimental study in the auditory and visual cortices by Ji et. al. \cite{Ji2015} supports such a bias in the number of feedforward inputs to excitatory and inhibitory populations. This is a prediction that has to be further corroborated in future experiments. \\

In summary, the fluctuations in the neuronal input and trial-to-trial variability is either increased or decreased depending on the type and quantity of excess bidirectionality. Crucially, these effects occur only as a result of the network operating in the balanced regime. Excess bidirectionality does not significantly alter the functional properties of the network. Thus, fine structure can significantly change the dynamics of balanced networks without impeding its function. Future work will have to study the role of other more complex types of fine structures and their effects. 
\section{Methods}
\textit{Model of rodent layer 2/3 with excess bidirectionality}
The neurons in the Layer 2/3 of rodents show strong orientation selectivity (OS) already at eye opening. Neurons in this layer are arranged in a salt and pepper fashion so that each neurons integrated inputs from neurons of all preferred orientations(PO) (i.e the rodent V1 lacks an orientation map or functional architecture). To investigate the effect of bidirectionality on the spiking irregularity and functional properties of the cortex, we used a modified version of a conductance based spiking model of rodent layer 2/3 developed in Hansel and Vreeswijk 2012 \cite{Hansel2012}. They showed that strong OS does not require a functional architecture, provided that the cortex is operating in the balaned regime.
The neurons in the model were arranged on two dimensional surface where the connection probabilities between neurons were only dependent on anatomical distance within the surface. L4 neurons were assumed to be OS and L2/3 neurons received feedforward inputs from randomly selected L4 neurons with different POs. Hence the total input that each L2/3 neurons receives has a large untuned component and a comparably weak tuned part. In the model, the recurrent dynamics of the network is such that the total inhibitory and excitatory currents cancel each other. Hence, the large untuned component is dynamically suppressed and the weakly tuned input is amplified by the single neuron threshold rendering the neurons in the network OS. \\

\textbf{\textit{Single neuron dynamics:}}
The single neuron dynamics are described by an one compartment conductance based model with sodium and potassium currents responsible for spike generation \cite{wang1996}. The excitatory neurons have an additional adaptation current.

\begin{equation}
\begin{split}
C_{m} \frac{dV_{i}^{A}}{dt} = -I_{L, i}^{A} -I_{Na, i}^{A}-I_{K, i}^{A} -I_{adapt, i}^{A}\  \\ + I_{rec, i}^{A} + I_{ff, i}^{A} \,\,\;\; (A = E, I) 
\end{split}
\end{equation}

\begin{align}
I_{L, i}^{A} &= g_{L}^{A} (V_{i}^{A} - V_{L}) \\
I_{Na, i}^{A} &= g_{Na}^{A} m_{\infty}^{3} h (V_{i}^{A} - V_{Na}) \\
I_{K, i}^{A} &= g_{K}^{A} n^{4} (V_{i}^{A} - V_{K}) \\
I_{adapt, i}^{A} &= g_{adapt}^{A} z (V_{i}^{A} - V_{K}) \\
I_{rec, i}^{A} &= \sum_{B} g_{i}^{AB} (V_{i}^{A} - V_{B}) \\
I_{rec, i}^{A} &= \sum_{B} g_{i}^{AB} (V_{i}^{A} - V_{B}) \\
g_{i}^{AB} &= \bar{g}^{AB} \sum_{j} C^{AB}_{ij} \sum_{k} S(t - t^{B}_{j, k}) \\
I_{ff, i}^{A} &= g_{ff, i}^{A} (\theta, t) (V_{i}^{A} - V_{E})
\end{align}

\begin{equation} 
\begin{split}
& g_{ff, i}^{A} (\theta, t)  = \frac{\bar{g}^{A}_{ff}}{\tau_{syn}} \;\; \times \\
& \int\limits_{-\infty}^{t}  \left( R_{i, tot}^{A} (\theta, t') + \sqrt{R_{i, tot}^{A} (\theta, t')}  \eta^{A}_{i}(t) \right) e^{-(t' - t) /\tau_{syn}} dt' 
\end{split}
\end{equation}

%\begin{equation}
%\begin{split}
%%& R^{A}_{i, tot} (\theta, t) &= c_{ff} K \left[ R_{0}^{ff} + R_{1}^{ff}(C)] \right] + \sqrt{c_{ff} K} R_{0}^{ff} x_{i}^{A} + \sqrt{c_{ff} K} R_{1}^{ff}(C) x \\
%%& \left[x_i^A + \xi_A z^A_{1, i} cos2(\theta - \Delta_i^A) + \mu_A z^A_{2, i} cos(wt - \phi^A_{1, i}) + \frac{\xi_A \mu_A}{2} \{ z^A_{3, i}cos(2\theta + wt - \phi^A_{2, i}) + z^A_{4, i}cos(2\theta - wt + \phi^A_{3, i}) } \right]
%\end{split}
%\end{equation}


%g_{ff, i}^{A} (\theta, t) &= \frac{\bar{g}^{A}_{ff}}{\tau_{syn}} K_{ff} \left( R_{i}^{A} (\theta) + \sqrt{\frac{R_{i}^{A} (\theta)}{K_{ff}}} \eta^{A}_{i}(t) \right)
%  g_{ff, i}^{A} (\theta, t) &= \frac{\bar{g}^{A}_{ff}}{\tau_{syn}} \int_{-\infty}^{t} \left( R_{i}^{A} (\theta) + \sqrt{\frac{R_{i}^{A} (\theta)}{K_{ff}}} \eta^{A}_{i}(t) e^{-(t' - t) /\tau_{syn}} dt' \right)
%I_{rec, i}^{A} &= \sum_{B} g_{i}^{AB} (V_{i}^{A} - V_{B})
%, \;\; g_{i}^{AB} = \bar{g}^{AB} \sum_{j} C^{AB}_{ij} \sum_{k}S(t - t^{B}_{j, k}) 
%I_{ff, i}^{A} &= g_{ff, i}^{A} (\theta, t) (V_{i}^{A} - V_{E}) 

\begin{align}
\overline{g}_{AB} &= \frac{G_{AB}}{\sqrt{K}} \\
\overline{g}^{A}_{ff} &= \frac{G^{A}_{ff}}{c_{ff} \sqrt{K}} \\
c_{ff} &= \frac{K_{ff}}{K}
\end{align}
where, \\
$\xi_A:$ Tuning amplitude of input from L4 to L2/3. \\
$x^A_i, z^A_i , \Delta^A_i , \nu^A_i:$ Random variables. \\
$\theta:$ Stimulus orientation. \\  
$C:$ Stimulus contrast. \\
$R_0^{ff}:$ Background firing rate. \\
$R_1^{ff}:$ Visual response in L4. \\

\textbf{\textit{Parameters used:}} 
$g_{Na}=100mS / cm^2$, $V_{Na} = 55mV$, 
$g_{K} = 40mS / cm^2$, $V_{K} = -80mV$,
$V_L = -65mV$, 
$C_m = 1 \mu F / cm^2$, 
$g_L = 0.1 mS / cm^2$. 
Only excitatory neurons had adaptation current with $g_{adapt} = 0.5 mS/cm^2$. The synaptic time constant $\tau_{syn}$ was set to $3ms$. $G_{EE} = 0.15$, $G_{IE} = 0.45$, $G_{EI} = 2.0$, $G_{II} = 3 \; ms \; mS / cm^2$. 
$\xi_A = 0.8$, $C = 100$, $R_0^{ff} = 2Hz$, $R_1^{ff} = 20Hz$. 
$K = 500, \; K_{ff}^{E} = 100, \; K_{ff}^{I} = 800$. \\

\textbf{\textit{Generating excess bidirectionality in the connectivity matrix:}}
%In an Erdös-Rényi network the number of bidirectional network is $K^{2}/N$ on average
To generate the connectivity matrix with an excess bidirectionality of $p$, a neuron $i$ from population $A$ and neuron $j$ from population $B$  were connected reciprocally with a  probability of $p_{ij}^{AB} = p * \frac{K}{N_{B}} + (1 - p) * \frac{K^2}{N_{B}^2}$. 
Unidirectional connections were  made with a probablity $p_{ij}^{AB}  = 2 * (1 - p) * \frac{K}{N_{B}} * (1 -  \frac{K}{N_{B}})$. This gives a connectivity matrix with each neuron receiving $K$ inputs on average with $pK$ number of bidirectional connections. Whereas, a random network has $\frac{K^2}{N}$ bidirectional connections on average. \\

\textbf{\textit{Fano factor:}}
Fano factor(FF) for a neuron is defined as,\\
\begin{equation}
FF = \frac{Var[N_{k}]}{E[N_{k}]},
\end{equation}
where $N_{k}$ is the number of spikes in trial $k$. \\
We repeated the simulation with different initial conditions while keeping the input stimulus fixed. The Fano factor was then determined for all neurons by computing the mean spike count and spike count variance upon repeated stimulus presentation over hundred simulated trials. \\

\textbf{\textit{Autocorrelation (AC):}}
Given a spike train $S(t) = \sum_{k} \delta(t - t^k)$, the 
autocorrelation function is defined as, \\
\begin{equation}
C(\tau) = \left\langle S(t) S(t + \tau) \right\rangle _ {t},
\end{equation}
where $\left\langle \cdot \right\rangle _ {t}$ is the average over time. 
We binned the spike train in $\Delta t = 1ms$ bins. Let the spike count in the $n^{th}$ bin be $N_{i}(n)$. The population averaged autocorrelation function for the was defined as, \\
\begin{align}
AC(\tau) &= \left[ \frac{\langle N_{i}(t)N_{i}(t + \tau) \rangle_t}{\Delta t \; T \; r_{i}} \right]_i
\end{align}
where,\\
\begin{equation}
\tau = n \Delta t,\;\; t = m \Delta t
\end{equation}
$T$ is the duration of simulation in seconds.\\
$r_i$ is the mean firing rate of the $i^{th}$ neuron. \\
$\left[ \cdot \right]_i$ is the average over the population. \\
The peak at zero was removed and the AC normalization is such that at long time lags the AC function of individual neurons converge to their respective mean activity. \\

%$$\left\langle f(t) \right\rangle _ {t} = \lim_{T \rightarrow \infty} \int_{-T/2}^{T/2} dt f(t)$$

\textbf{\textit{Coefficient of variation ($CV$):}}
Given a spike train with $N$ spikes occurring at times $t_{i}$, the ISIs are given by, \\
\begin{align}
\Delta t_{i} &= t_{i} - t_{i-1}
\end{align}
%CV = \frac{\sigma_{ISI}}{\mu_{ISI}}
$CV$ is defined as: \\
\begin{align}
CV &= \frac{\sqrt{Var[\Delta t_{i}]}}{E[\Delta t_{i}]}
\end{align}

For a renewal process, FF is given by, 
\begin{align}
FF = CV^{2}
\end{align}
and for a stationary non-renewal process, 
\begin{align}
FF = CV^{2} (1 + 2 \sum_{i} SRC_{i})
\label{ffcveq}
\end{align}
where $SRC_I$ is the Spearman rank order correlation coefficient of order $i$. It is computed by replacing each $ISI$ with its rank. It is a measure of serial correlations in the spike trains. Positive serial correlations increase FF and negative serial correlations reduce FF.   \\

\textbf{\textit{Coefficient of variation 2 ($CV_2$):}}
If a regular spike train has a slowly modulated firing rate, the CV obtained will be high even though the spike train is regular. To overcome this problem another measure, $CV_2$,  is usually adopted to quantify the intrinsic variability \cite{Holt1996}.

$CV_2$ for the spike train is defined as: \\
\begin{align}
CV_{2} &= \left\langle 2 \frac{|\Delta t_{i+1} - \Delta t_{i}|}{\Delta t_{i+1} + \Delta t_{i}} \right\rangle_{i}
\end{align}
where $\left\langle \cdot \right\rangle_{i}$ stands for averaging over all the $N$ spikes. \\

\textbf{\textit{Tuning Properties:}}
The Circular variance( = 1 - OSI) for the $k^{th}$ cell was defined as: $CircVar_{k} = 1 - \frac{|z_k|}{\sum_i r_k(i) }$\\ where,\\
 $z_k = \sum_i r_k(i)*exp(2*j*\theta_i)$;    $j = \sqrt{-1}$\\
 $\theta_i = 2i\pi/8$;    $i \in \{0,1, ..., 7\}$\\
 $r_k(i)$ is the firing rate of the $k^{th}$ cell at stimulus orientation $\theta_{i}$.\\


\newpage
\begin{small}
\bibliography{references}
\bibliographystyle{unsrt}
\end{small}

\newpage
\begin{figure*}
\centering
\includegraphics[scale=1]{./figs/INKSCAPE/control_props.pdf}
\caption{\small{Properties of the model. $N_E = N_I = 20000, K = 500, K_{ff}^E = 100, K_{ff}^I = 800, \tau_s = 3ms, \xi = 0.8$. (a) Sample voltage trace in a neuron in the network. (b) Total Excitatory(in black) current and Inhibitory(in red) currents. The mean net input(in blue) is zero with fluctuations of O(Threshold). (c) Example tuning curves for one E and I neuron. Firing rate and circular variance distributions in (d) and (e) respectively.}}
\label{fig:control}
\end{figure*}



\begin{figure*}
\centering
\includegraphics[scale=1]{./figs/INKSCAPE/E2E_1_v2.pdf}
\caption{\small{Functional properties do not change with bidirectionality in E-to-E. $N_E = N_I = 20000, K = 500, K_{ff}^E = 100, K_{ff}^I = 800, \tau_s = 3ms, \xi = 0.8$. Example tuning curves for (a) p = 0, (b) p = 0.8. Firing rate distributions for (c) E, (d) I populations. Circular variance distributions for (e) E (f) I populations.}}
\label{fig:e2e1}
\end{figure*}

\begin{figure*}
\centering
\includegraphics[scale=1]{./figs/INKSCAPE/e2e_0_v1.pdf}
\caption{\small{Bidirectionality in E-to-E has negligible effect on spiking irregularity and fluctuations. $N_E = N_I = 20000, K = 500, K_{ff}^E = 100, K_{ff}^I = 800, \tau_s = 3ms, \xi = 0.8$. Population averaged autocorrelation functions for (a) E, (b) I populations. Fano factor distributions for (c) E, (d) I populations. (e) $CV$ and (f) $CV_{2}$ distributions for E and I in top and bottom panels respectively.}}
\label{fig:e2e0}
\end{figure*}

\begin{figure*}
\centering
\includegraphics[scale=1]{./figs/INKSCAPE/I2I_1_v2.pdf}
\caption{\small{Strong bidirectionality in I-to-I leads to broadening of firing rate distribution and sharpens the tuning curves of I cells. $N_E = N_I = 20000, K = 500, K_{ff}^E = 100, K_{ff}^I = 800, \tau_s = 3ms, \xi = 0.8$. Example tuning curves for (a) p = 0.5, (b) p = 0.8. Firing rate distributions for (c) E, (d) I populations. Circular variance distributions for (e) E (f) I populations.}}
\label{fig:i2i1}
\end{figure*}        
     
\begin{figure*}
\centering
\includegraphics[scale=1]{./figs/INKSCAPE/I2I_0_v1.pdf}
\caption{\small{Bidirectionality in I-to-I slows down fluctuations and increases response variability. $N_E = N_I = 20000, K = 500, K_{ff}^E = 100, K_{ff}^I = 800, \tau_s = 3ms, \xi = 0.8$. Population averaged autocorrelation functions for (a) E, (b) I populations. (c) The decorrelation time estimated by an exponential fit. (d) $CV$ and $CV_{2}$ distributions for E and I in top and bottom panels respectively. Fano factor distributions for (e) E, (f) I populations. (g) Example voltage trace at p = 0.8.}}
\label{fig:i2i0}
\end{figure*}

\begin{figure*}
	\centering
\includegraphics[scale=1]{./figs/INKSCAPE/slow_syn_0_v1.pdf}
\caption{\small{Slow synapses with I-to-I bidirectionality. $N_E = N_I = 20000, K = 500, K_{ff}^E = 100, K_{ff}^I = 800, \xi = 0.8$. Estimated decorrelation times are well approximated by a power law (a) E (b) I. Population averaged Fano factor for (a) E (b) I. The steady state firing rates of neurons for two different initial conditions for (e) Control, $\tau_s = 48ms$ (f) $p = 0.9$, , $\tau_s = 48ms$}}
\label{fig:slowsyn0}
\end{figure*}         
 
\begin{figure*}
\centering
\includegraphics[scale=1]{./figs/INKSCAPE/E2I_1_v2.pdf}
\caption{\small{Bidirectionality in E-to-I. $N_E = N_I = 20000, K = 500, K_{ff}^E = 100, K_{ff}^I = 800, \tau_s = 3ms, \xi = 0.8$. Example tuning curves for (a) p = 0.5, (b) p = 0.8. Firing rate distributions for (c) E, (d) I populations. Circular variance distributions for (e) E (f) I populations.}}
\label{fig:e2i1}
\end{figure*}      

\begin{figure*}
\centering
\includegraphics[scale=1]{./figs/INKSCAPE/e2i_0_v1.pdf}
\caption{\small{Bidirectionality in E-to-I leads to rapid deccorelation and reduced response variability. $N_E = N_I = 20000, K = 500, K_{ff}^E = 100, K_{ff}^I = 800, \tau_s = 3ms, \xi = 0.8$. Population averaged autocorrelation functions for (a) E, (b) I populations. Fano factor distributions for (c) E, (d) I populations. (e) $CV$ and (f) $CV_{2}$ distributions for E and I in top and bottom panels respectively.}}
\label{fig:e2i0}
\end{figure*}         

%\begin{figure*}
%\centering
%\includegraphics[scale=1.0]{./figs/INKSCAPE/vmtrace_v1.pdf}
%\caption{\small{Example voltage traces showing increased burstiness with increasing $p$ and $\tau_s$ (a) $p=0, \; 28.9Hz$, (b) $p = 0, \; \tau_{s} = 12ms, \; 33.0Hz$, (c) $p = 0.8, \; \tau_{s} = 3ms, \; 26.1Hz$,(d) $p = 0.8, \; \tau_{s} = 6ms, \; 25.1Hz$, (e) $p = 0.8, \; \tau_{s} = 12ms$}}
%\label{fig:vmtrace}
%\end{figure*}
%
%\begin{figure*}
%\centering
%\includegraphics[scale=1.0]{./figs/INKSCAPE/cur_traces.pdf}
%\caption{\small{Example current traces onto a neuron. Total excitatory and inhibitory currents in black and red respectively, and their sum in blue. Top panel is for Control and bottom for $p = 0.8$}}
%\label{fig:curtrace}
%\end{figure*}             

\end{document}