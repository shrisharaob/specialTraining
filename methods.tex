\section{Methods}
\textit{Model of rodent layer 2/3 with excess bidirectionality}
The neurons in the Layer 2/3 of rodents show strong orientation selectivity (OS) already at eye opening. Neurons in this layer are arranged in a salt and pepper fashion so that each neurons integrated inputs from neurons of all preferred orientations(PO) (i.e the rodent V1 lacks an orientation map or functional architecture). To investigate the effect of bidirectionality on the spiking irregularity and functional properties of the cortex, we used a modified version of a conductance based spiking model of rodent layer 2/3 developed in Hansel and Vreeswijk 2012 \cite{Hansel2012}. They showed that strong OS does not require a functional architecture, provided that the cortex is operating in the balaned regime.

\textbf{\textit{Single neuron dynamics:}}
The single neuron dynamics are described by an one compartment conductance based model with sodium and potassium currents responsible for spike generation \cite{wang1996}. The excitatory neurons have an additional adaptation current.

\begin{equation}
\begin{split}
C_{m} \frac{dV_{i}^{A}}{dt} = -I_{L, i}^{A} -I_{Na, i}^{A}-I_{K, i}^{A} -I_{adapt, i}^{A}\  \\ + I_{rec, i}^{A} + I_{ff, i}^{A} \,\,\;\; (A = E, I) 
\end{split}
\end{equation}

\begin{align}
I_{L, i}^{A} &= g_{L}^{A} (V_{i}^{A} - V_{L}) \\
I_{Na, i}^{A} &= g_{Na}^{A} m_{\infty}^{3} h (V_{i}^{A} - V_{Na}) \\
I_{K, i}^{A} &= g_{K}^{A} n^{4} (V_{i}^{A} - V_{K}) \\
I_{adapt, i}^{A} &= g_{adapt}^{A} z (V_{i}^{A} - V_{K}) \\
I_{rec, i}^{A} &= \sum_{B} g_{i}^{AB} (V_{i}^{A} - V_{B}) \\
I_{rec, i}^{A} &= \sum_{B} g_{i}^{AB} (V_{i}^{A} - V_{B}) \\
g_{i}^{AB} &= \bar{g}^{AB} \sum_{j} C^{AB}_{ij} \sum_{k} S(t - t^{B}_{j, k}) \\
I_{ff, i}^{A} &= g_{ff, i}^{A} (\theta, t) (V_{i}^{A} - V_{E})
\end{align}

\begin{equation} 
\begin{split}
& g_{ff, i}^{A} (\theta, t)  = \frac{\bar{g}^{A}_{ff}}{\tau_{syn}} \;\; \times \\
& \int\limits_{-\infty}^{t}  \left( R_{i, tot}^{A} (\theta, t') + \sqrt{R_{i, tot}^{A} (\theta, t')}  \eta^{A}_{i}(t) \right) e^{-(t' - t) /\tau_{syn}} dt' 
\end{split}
\end{equation}

%\begin{equation}
%\begin{split}
%%& R^{A}_{i, tot} (\theta, t) &= c_{ff} K \left[ R_{0}^{ff} + R_{1}^{ff}(C)] \right] + \sqrt{c_{ff} K} R_{0}^{ff} x_{i}^{A} + \sqrt{c_{ff} K} R_{1}^{ff}(C) x \\
%%& \left[x_i^A + \xi_A z^A_{1, i} cos2(\theta - \Delta_i^A) + \mu_A z^A_{2, i} cos(wt - \phi^A_{1, i}) + \frac{\xi_A \mu_A}{2} \{ z^A_{3, i}cos(2\theta + wt - \phi^A_{2, i}) + z^A_{4, i}cos(2\theta - wt + \phi^A_{3, i}) } \right]
%\end{split}
%\end{equation}


%g_{ff, i}^{A} (\theta, t) &= \frac{\bar{g}^{A}_{ff}}{\tau_{syn}} K_{ff} \left( R_{i}^{A} (\theta) + \sqrt{\frac{R_{i}^{A} (\theta)}{K_{ff}}} \eta^{A}_{i}(t) \right)
%  g_{ff, i}^{A} (\theta, t) &= \frac{\bar{g}^{A}_{ff}}{\tau_{syn}} \int_{-\infty}^{t} \left( R_{i}^{A} (\theta) + \sqrt{\frac{R_{i}^{A} (\theta)}{K_{ff}}} \eta^{A}_{i}(t) e^{-(t' - t) /\tau_{syn}} dt' \right)
%I_{rec, i}^{A} &= \sum_{B} g_{i}^{AB} (V_{i}^{A} - V_{B})
%, \;\; g_{i}^{AB} = \bar{g}^{AB} \sum_{j} C^{AB}_{ij} \sum_{k}S(t - t^{B}_{j, k}) 
%I_{ff, i}^{A} &= g_{ff, i}^{A} (\theta, t) (V_{i}^{A} - V_{E}) 

\begin{align}
\overline{g}_{AB} &= \frac{G_{AB}}{\sqrt{K}} \\
\overline{g}^{A}_{ff} &= \frac{G^{A}_{ff}}{c_{ff} \sqrt{K}} \\
c_{ff} &= \frac{K_{ff}}{K}
\end{align}
where, \\
$\xi_A:$ Tuning amplitude of input from L4 to L2/3. \\
$x^A_i, z^A_i , \Delta^A_i , \nu^A_i:$ Random variables. \\
$\theta:$ Stimulus orientation. \\  
$C:$ Stimulus contrast. \\
$R_0^{ff}:$ Background firing rate. \\
$R_1^{ff}:$ Visual response in L4. \\

\textbf{\textit{Parameters used:}} 
$g_{Na}=100mS / cm^2$, $V_{Na} = 55mV$, 
$g_{K} = 40mS / cm^2$, $V_{K} = -80mV$,
$V_L = -65mV$, 
$C_m = 1 \mu F / cm^2$, 
$g_L = 0.1 mS / cm^2$. 
Only excitatory neurons had adaptation current with $g_{adapt} = 0.5 mS/cm^2$. The synaptic time constant $\tau_{syn}$ was set to $3ms$. $G_{EE} = 0.15$, $G_{IE} = 0.45$, $G_{EI} = 2.0$, $G_{II} = 3 \; ms \; mS / cm^2$. 
$\xi_A = 0.8$, $C = 100$, $R_0^{ff} = 2Hz$, $R_1^{ff} = 20Hz$. \\

\textbf{\textit{Generating excess bidirectionality in the connectivity matrix:}}
%In an Erdös-Rényi network the number of bidirectional network is $K^{2}/N$ on average
To generate the connectivity matrix with an excess bidirectionality of $p$, a neuron $i$ from population $A$ and neuron $j$ from population $B$  were connected reciprocally with a  probability of $p_{ij}^{AB} = p * \frac{K}{N_{B}} + (1 - p) * \frac{K^2}{N_{B}^2}$. 
Unidirectional connections were  made with a probablity $p_{ij}^{AB}  = 2 * (1 - p) * \frac{K}{N_{B}} * (1 -  \frac{K}{N_{B}})$. This gives a connectivity matrix with each neuron receiving $K$ inputs on average with $pK$ number of bidirectional connections. Whereas, a random network has $\frac{K^2}{N}$ bidirectional connections on average. \\

\textbf{\textit{Fano factor:}}
Fano factor(FF) for a neuron is defined as,\\
\begin{equation}
FF = \frac{Var[N_{k}]}{E[N_{k}]},
\end{equation}
where $N_{k}$ is the number of spikes in trial $k$. \\
Fano factor was determined for all neurons by computing the mean spike count and spike count variance upon repeated stimulus presentation over hundred simulated trials. \\

\textbf{\textit{Autocorrelation (AC):}}
Given a spike train $S(t) = \sum_{k} \delta(t - t^k)$, the 
autocorrelation function is defined as, \\
\begin{equation}
C(\tau) = \left\langle S(t) S(t + \tau) \right\rangle _ {t},
\end{equation}
where $\left\langle \cdot \right\rangle _ {t}$ is the average over time. 
We binned the spike train in $\Delta t = 1ms$ bins. Let the spike count in the $n^{th}$ bin be $N_{i}(n)$. The population averaged autocorrelation function for the was defined as, \\
\begin{align}
AC(\tau) &= \left[ \frac{\langle N_{i}(t)N_{i}(t + \tau) \rangle_t}{\Delta t \; T \; r_{i}} \right]_i
\end{align}
where,\\
\begin{equation}
\tau = n \Delta t,\;\; t = m \Delta t
\end{equation}
$T$ is the duration of simulation in seconds.\\
$r_i$ is the mean firing rate of the $i^{th}$ neuron. \\
$\left[ \cdot \right]_i$ is the average over the population. \\
The peak at zero was removed and the AC normalization is such that at long time lags the AC function of individual neurons converge to their respective mean activity. \\

%$$\left\langle f(t) \right\rangle _ {t} = \lim_{T \rightarrow \infty} \int_{-T/2}^{T/2} dt f(t)$$

\textbf{\textit{Coefficient of variation ($CV$):}}
Given a spike train with $N$ spikes occurring at times $t_{i}$, the ISIs are given by, \\
\begin{align}
\Delta t_{i} &= t_{i} - t_{i-1}
\end{align}
%CV = \frac{\sigma_{ISI}}{\mu_{ISI}}
$CV$ is defined as: \\
\begin{align}
CV &= \frac{\sqrt{Var[\Delta t_{i}]}}{E[\Delta t_{i}]}
\end{align}

For a renewal process, FF is given by, 
\begin{align}
FF = CV^{2}
\end{align}
and for a stationary non-renewal process, 
\begin{align}
FF = CV^{2} (1 + 2 \sum_{i} SRC_{i})
\label{ffcveq}
\end{align}
where $SRC_I$ is the Spearman rank order correlation coefficient of order $i$. It is computed by replacing each $ISI$ with its rank. It is a measure of serial correlations in the spike trains. Positive serial correlations increase FF and negative serial correlations reduce FF.   \\

\textbf{\textit{Coefficient of variation 2 ($CV_2$):}}
If a regular spike train has a slowly modulated firing rate, the CV obtained will be high even though the spike train is regular. To overcome this problem another measure, $CV_2$,  is usually adopted to quantify the intrinsic variability \cite{Holt1996}.

$CV_2$ for the spike train is defined as: \\
\begin{align}
CV_{2} &= \left\langle 2 \frac{|\Delta t_{i+1} - \Delta t_{i}|}{\Delta t_{i+1} + \Delta t_{i}} \right\rangle_{i}
\end{align}
where $\left\langle \cdot \right\rangle_{i}$ stands for averaging over all the $N$ spikes. \\

\textbf{\textit{Tuning Properties:}}
The Circular variance( = 1 - OSI) for the $k^{th}$ cell was defined as: $CircVar_{k} = 1 - \frac{|z_k|}{\sum_i r_k(i) }$\\ where,\\
 $z_k = \sum_i r_k(i)*exp(2*j*\theta_i)$;    $j = \sqrt{-1}$\\
 $\theta_i = 2i\pi/8$;    $i \in \{0,1, ..., 7\}$\\
 $r_k(i)$ is the firing rate of the $k^{th}$ cell at stimulus orientation $\theta_{i}$.\\
