\section{Discussion}


Both the excitatory and inhibitory populations exhibit strong OS. The inhibitory population in the model had tuning properties very similar to the excitatory population. However, there is a growing consensus that PV\textsuperscript{+} cells in rodents are more broadly tuned than excitatory neurons \cite{niell2008highly, kerlin2010broadly, liu2009visual, sohya2007gabaergic, kameyama2010difference} (also see \cite{Runyan2010, Ma2010}). Here we show that this can be explained when the inhibitory neurons(PV\textsuperscript{+}) in L2/3 receive more numerous and weaker synapses from L4 than the excitatory population.  We discovered that, it is possible to explain broadly tuned inhibitory cells by incorporating more numerous and weaker feed-forward input to the inhibitory population. In the simulations, it was necessary to scale the synaptic strengths appropriately as the number of feedforward inputs increases to ensure that the mean population activities remain the same(Methods). Since the mean activities were fixed, broadening of the tuning of inhibitory population was not due to a change in firing rates. Recent experimental study in the auditory and visual cortices by Ji et. al. \cite{Ji2015} supports such a bias in the number of feedforward inputs to excitatory and inhibitory populations. This is a prediction that has to be further corroborated in future experiments. In all our simulations, we assumed that the inhibitory population in L2/3 receives weaker and more numerous inputs than the excitatory population($K_{ff}^{E} = 100, \; K_{ff}^{I} = 800$). This gives us mean values of Circular variance($CircVar = 1 - OSI$) that are comparable to the experimental numbers. We studied the effects of excess bidirectionality with these parameters and characterized the spiking irregularity, fluctuations and, response properties of the our L2/3 network model.  \\


We show that depending on the type of bidirectional connections in balanced networks, the input decorrelation times either increase or decrease compared to such networks with random connectivity. Bidirectionality in I-to-I affects the time scale of the dynamics only because the network is operating in the balanced regime. This could be a mechanism by which slow fluctuations are generated. As noted earlier, there is significant in the power spectrum of membrane voltages recorded from the cortex. But, there is very little experimental data reporting significant I-to-I bidirectionality. Future experiments might shed more light on the connectivity scheme within the inhibitory populations. Remarkably, E-to-E bidirectionality which has been experimentally documented has negligible effects in the balanced V1 model. This is not surprising, considering that the balanced networks require that inhibitory population have stronger recurrent synapses. While excitatory neurons have weaker recurrent synapses compared to the inhibitory neurons. \\
Slow synapses and I-to-I bidrectionality leads to multistability(Fig. \ref{fig:slowsyn0}f). The emergence of multistability has been studied in Spin Glasses. A fully bidirectional network is very similar to a diluted spin glass\cite{megard1987spin}. At low temperatures, Replica Symmetry Breaking (RSB) may be expected at the transition between ferromagnetic and spin glass phases. Therefore, as the level of bi-directionality is increased one can expect glassy behaviour to appear with a slowing down of the fluctuations\cite{Crisanti1987, Crisanti1988}. The time scales of these fluctuations are typically well approximated by a power law. Although, the connectivity in SG models do not obey Dales law, one could draw an analogy between Temperature and synaptic time constant. Thus, lowering the temperature can be seen as increasing the synaptic time constant. The decorrelations times estimated from the simulations also show similar power law behaviour. (Fig. \ref{fig:slowsyn0}a \& Fig. \ref{fig:slowsyn0}b) \\
In summary, the fluctuations in the neuronal input and trial-to-trial variability is either increased or decreased depending on the type and quantity of excess bidirectionality. Excess bidirectionality does not significantly alter the functional properties of the network. Thus, fine structure can significantly change the dynamics of balanced networks without impeding its function. 
