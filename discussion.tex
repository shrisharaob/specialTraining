\section{Discussion}
%\begin{itemize}
%\item Summarize results and emphasize that Balance is necessary for there to be any effect at all due to excess bidirectionality.
%\item E-to-E biderctionality evident in experiments, has negligible effects on tuning and response properties in the V1 model.
%\item On the other hand, I-to-I bidirectionality is much more effective in slowing down the dynamics and must be investigated in experiments.  
%\item Slow synapses amplify the effect of bidirectionality
%\end{itemize}

%We have shown that above chance levels of reciprocity leads to slowing down of dynamics. It is important to note that slowing down occurs only because the network is operating in a balanced regime. \\
%Surprisingly, the introducing E-to-E reciprocity in the V1 model produces no significant changes in the firing rates,  auto-correlation functions, tuning properties and Fano factors. Whereas, I-to-I reciprocity leads to a slowing down of the dynamics. At reasonable levels of excess bidirectionality, the firing rates and tuning properties remain unchanged. \\
%These results suggest that the local circuitry of the inhibitory population plays a crucial role in shaping the dynamics of the cortex. Thus it is essential that future experiments investigate the structure of local inhibitory circuits and their electro-physiological properties.  \\

% RIGHT AMOUNT OF e-TO-i ALONG WITH I-TO-I COULD LEAD TO NOT EFFECT AT ALL

We show that depending on the type of bidirectional connections in balanced networks, the input decorrelation times either increase or decrease compared to such networks with random connectivity. Bidirectionality in I-to-I affects the time scale of the dynamics only because the network is operating in the balanced regime. This could be a mechanism by which slow fluctuations are generated. As noted earlier, there is significant in the power spectrum of membrane voltages recorded from the cortex. But, there is very little experimental data reporting significant I-to-I bidirectionality. Future experiments might shed more light on the connectivity scheme within the inhibitory populations. Remarkably, E-to-E bidirectionality which has been experimentally documented has negligible effects in the balanced V1 model. This is not surprising, considering that the balanced networks require that inhibitory population have stronger recurrent synapses. While excitatory neurons have weaker recurrent synapses compared to the inhibitory neurons. \\
Slow synapses and I-to-I bidrectionality leads to multistability(Fig. \ref{fig:slowsyn0}f). The emergence of multistability has been studied in Spin Glasses. A fully bidirectional network is very similar to a diluted spin glass\cite{megard1987spin}. At low temperatures, Replica Symmetry Breaking (RSB) may be expected at the transition between ferromagnetic and spin glass phases. Therefore, as the level of bi-directionality is increased one can expect glassy behaviour to appear with a slowing down of the fluctuations\cite{Crisanti1987, Crisanti1988}. The time scales of these fluctuations are typically well approximated by a power law. Although, the connectivity in SG models do not obey Dales law, one could draw an analogy between Temperature and synaptic time constant. Thus, lowering the temperature can be seen as increasing the synaptic time constant. The decorrelations times estimated from the simulations also show similar power law behaviour. (Fig. \ref{fig:slowsyn0}a \& Fig. \ref{fig:slowsyn0}b) \\
In summary, the fluctuations in the neuronal input and trial-to-trial variability is either increased or decreased depending on the type and quantity of excess bidirectionality. Excess bidirectionality does not significantly alter the functional properties of the network. Thus, fine structure can significantly change the dynamics of balanced networks without impeding its function. 

%In the simulations this is seen in the presence of slow synapses at high levels of bidirectionality (Fig. \ref{fig:slowsyn0}f). Slower synapses 



%However, to what degree does the decorrelation times change and what other response properties are significantly affected? To address this question we use numerical simulations of a conductance based model of layer 2/3 rodent V1. Excess bidirectionality in the E results in negligible effects. On the other hand, high levels of bidirectional connections in I-to-I results in slowing down of fluctuations and an increase in the trial-to-trial variability. This could be a possible mechanism which leads to significant low frequency components observed in the power spectrum of membrane voltages recorded in the cortex. But excess bidirectionality in I has little experimental evidence. Contrary to the increase in decorrelation times with excess bidirectionality in I, excess bidirectionality between E and I results in its reduction. (Fig. \ref{e2i}) 
%Consequently, the trial-to-trial variability is also reduced. \\



%A fully random network is characterized by a single ground state. Fig. \ref{fig:slowsyn0}e shows that steady state firing rates of all the neurons converge to nearby values even though the simulations are started with different initial conditions. On the other hand, a fully bidirectional network is very similar to a diluted spin glass. Thus a replica symmetry breaking (RSB) may be expected in this case. Therefore, as the level of bi-directionality is increased one can expect glassy behaviour to appear with a slowing down of the fluctuations. In the simulations this is seen in the presence of slow synapses at high levels of bidirectionality (Fig. \ref{fig:slowsyn0}f).





%EEG and LFP recordings usually have prominent 

