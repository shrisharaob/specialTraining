\section{Discussion}
We show that depending on the type of bidirectional connections in balanced networks, the input decorrelation time increases compared to networks with random connectivity. The effective net inputs that the neurons receive in a balanced network are of same order of magnitude as the threshold $\left( O(Threshold) \right)$. Modulations or changes in the inputs that are $O(Threshold)$ can lead to interesting network dynamics. In this study, we were interested in investigating if bidirectionality can lead to significant $O(Threshold)$ changes in the total input to neurons, and if so, to characterize the resulting changes in functional properties and network dynamics. In the standard balanced network the connectivity is assumed to be random. So, neurons receive weakly time-correlated order of threshold inputs or fast fluctuating inputs. As a result, input currents to the neurons in the network and consequently the spike time auto-correlations decorrelate rapidly\cite{carl1996, carl1998}. \\
%However, this is not consistent with what has been observed in cortical recordings, the power spectrum of membrane voltages recorded in the cortex show significant low frequency components. \cite{Tan2014}.
The presence of excess bidirectionality leads to formation of a significant number of loops in the network connectivity. Such loops can lead to non-negligible positive or negative feedback input currents if the effect is of $O(Threshold)$. $O(Threshold)$ changes lead to emergence of interesting properties in a balanced network. To see how such loops can contribute effects of $O(Threshold)$ let us consider one E neuron in a network with excess bidirectional probability $p$ where each neuron receives $K$ connections on average of strength $\frac{J_{EE}}{\sqrt{K}}$. Then, the expected number of bidirectional connections that this neuron receives is $pK$. When this neuron fires a spike, it results autocorrelation in an increase in the input to all the neurons it projects to by an amount proportional to $\frac{J_{EE}}{\sqrt{K}}$, thus increasing their probability of firing. And a fraction $p$ of those neurons project back to the neuron under consideration. This results in the neuron receiving an additional input proportional to $p J_{EE}^{2}$, which is $O(Threshold)$. Thus E-to-E excess bidirectionality should lead to a delayed positive self-coupling. 
%Therefore, presence of a fine structure such as bidirectionality in the network connectivity can lead to significant temporal correlations in the inputs.
%  Whereas, in a balanced network with random connectivity, the neuronal activities are weakly correlated resulting in all the neurons receiving weakly correlated inputs.\\ 
Remarkably, E-to-E bidirectionality which has been experimentally documented has negligible effects in the balanced V1 model. Why is there no effect even at excess E-to-E bidirectional probabilities close to one in our simulations?  This could be explained when we consider the balance conditions: $J_{EI} > J_{EE}$ and $\frac{J_{EI}}{J_{EE}} < \frac{J_{II}}{J_{IE}}$ \cite{carl1998}. The average E-to-E synaptic strength is smaller compared to that of average I-to-E and I-to-I, hence excess bidirectionality in E-to-E has negligible effect on the dynamics of the network. Furthermore, since I-to-I synaptic strength is strong, we hypothesized that the effects of excess bidirectionality might be more pronounced when it is introduced in the I population. So, we subsequently studied excess bidirectionality in the I population which indeed lead to longer decorrelation times.
%The slower dynamics in the network is due to temporally correlated inputs or slow fluctuations. 
The origin of these slow fluctuations is also due to delayed self-coupling, but mediated through an effective disinhibition. Self-coupling occurs due to the presence of loops in the I-to-I connectivity. Let us consider one I cell in the network, after it has fired a spike it hyperpolarizes its postsynaptic I neurons by a small amount. A fraction $pK$ of those neurons which project back to the neuron now have a slightly lower probability of firing. Hence, soon after the given neuron has spiked there is a significant reduction in the recurrent inhibition that it receives, thereby increasing the probability of its spiking soon after it has fired a spike. Therefore, the ISIs become positively correlated, such serial correlations in the spike trains leads to higher trial to trial variability. 

Bidirectionality in I-to-I affects the time scale of the dynamics only because the network is operating in the balanced regime. This could be a mechanism leading to slow fluctuations in the cortex as reported by significant power in the low frequency spectrum of membrane voltages\cite{Tan2014}. But, there is very little experimental data reporting significant I-to-I bidirectionality. Future experiments might shed more light on the connectivity scheme within the inhibitory populations.

Slow synapses and I-to-I bidrectionality leads to multistability(Fig. \ref{fig:slowsyn0}f). The emergence of multistability has been studied in Spin Glasses. A fully bidirectional network is very similar to a diluted spin glass\cite{megard1987spin}. At low temperatures, Replica Symmetry Breaking (RSB) may be expected at the transition between ferromagnetic and spin glass phases. Therefore, as the level of bi-directionality is increased one can expect glassy behaviour to appear with a slowing down of the fluctuations\cite{Crisanti1987, Crisanti1988}. The time scales of these fluctuations are typically well approximated by a power law. Although, the connectivity in SG models do not obey Dales law, one could draw an analogy between Temperature and synaptic time constant. Thus, lowering the temperature can be seen as increasing the synaptic time constant. The decorrelations times estimated from the simulations also show similar power law behaviour. (Fig. \ref{fig:slowsyn0}a \& Fig. \ref{fig:slowsyn0}b) \\

In the mechanism for OS in a random network proposed by Hansel and van Vreeswijk, both the excitatory and inhibitory populations exhibited strong OS. The inhibitory population in the model had tuning properties very similar to the excitatory population. However, there is a growing consensus that PV\textsuperscript{+} cells in rodents are more broadly tuned than excitatory neurons \cite{niell2008highly, kerlin2010broadly, liu2009visual, sohya2007gabaergic, kameyama2010difference} (also see \cite{Runyan2010, Ma2010}). Here we show that this can be explained when the inhibitory neurons in L2/3 receive more numerous and weaker synapses from L4 than the excitatory population.  We discovered that, it is possible to explain broadly tuned inhibitory cells by incorporating more numerous and weaker feed-forward input to the inhibitory population. In the simulations, it was necessary to scale the synaptic strengths appropriately as the number of feedforward inputs increases to ensure that the mean population activities remain the same. Since the mean activities were fixed, broadening of the tuning of inhibitory population was not due to a change in firing rates. Recent experimental study in the auditory and visual cortices by Ji et. al. \cite{Ji2015} supports such a bias in the number of feedforward inputs to excitatory and inhibitory populations. This is a prediction that has to be further corroborated in future experiments. \\

In summary, the fluctuations in the neuronal input and trial-to-trial variability is either increased or decreased depending on the type and quantity of excess bidirectionality. Crucially, these effects occur only as a result of the network operating in the balanced regime. Excess bidirectionality does not significantly alter the functional properties of the network. Thus, fine structure can significantly change the dynamics of balanced networks without impeding its function. Future work will have to study the role of other more complex types of fine structures and their effects. 