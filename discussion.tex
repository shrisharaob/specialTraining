\section{Discussion}
We show that depending on the type of bidirectional connections in balanced networks, the input decorrelation times either increase or decrease compared to such networks with random connectivity. Bidirectionality in I-to-I affects the time scale of the dynamics only because the network is operating in the balanced regime. This could be a mechanism by which slow fluctuations are generated. As noted earlier, there is significant in the power spectrum of membrane voltages recorded from the cortex. But, there is very little experimental data reporting significant I-to-I bidirectionality. Future experiments might shed more light on the connectivity scheme within the inhibitory populations. Remarkably, E-to-E bidirectionality which has been experimentally documented has negligible effects in the balanced V1 model. This is not surprising, considering that the balanced networks require that inhibitory population have stronger recurrent synapses. While excitatory neurons have weaker recurrent synapses compared to the inhibitory neurons. \\
Slow synapses and I-to-I bidrectionality leads to multistability(Fig. \ref{fig:slowsyn0}f). The emergence of multistability has been studied in Spin Glasses. A fully bidirectional network is very similar to a diluted spin glass\cite{megard1987spin}. At low temperatures, Replica Symmetry Breaking (RSB) may be expected at the transition between ferromagnetic and spin glass phases. Therefore, as the level of bi-directionality is increased one can expect glassy behaviour to appear with a slowing down of the fluctuations\cite{Crisanti1987, Crisanti1988}. The time scales of these fluctuations are typically well approximated by a power law. Although, the connectivity in SG models do not obey Dales law, one could draw an analogy between Temperature and synaptic time constant. Thus, lowering the temperature can be seen as increasing the synaptic time constant. The decorrelations times estimated from the simulations also show similar power law behaviour. (Fig. \ref{fig:slowsyn0}a \& Fig. \ref{fig:slowsyn0}b) \\
In summary, the fluctuations in the neuronal input and trial-to-trial variability is either increased or decreased depending on the type and quantity of excess bidirectionality. Excess bidirectionality does not significantly alter the functional properties of the network. Thus, fine structure can significantly change the dynamics of balanced networks without impeding its function. 
